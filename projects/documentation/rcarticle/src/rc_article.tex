%==============================================================================
% @author Clinton Freeman
% @date 12/18/2013
%==============================================================================

\documentclass{article}

% Package listing =============================================================

\usepackage{fullpage}
\usepackage{amsmath}
\usepackage{amssymb}
\usepackage{amsfonts}
\usepackage{amsthm}
\usepackage{lmodern}
\usepackage{algorithm}
\usepackage{algpseudocode}
%\usepackage{unicode-math} 
%\setmainfont[Ligatures=TeX,Scale=1.1]{Adobe Jenson Pro}
%\setsansfont{Frutiger LT Std}
%\setmathfont{XITS Math} 
%\setmathfont[range=\mathbb]{Linux Libertine}
\usepackage{hyperref}
%\usepackage{tabularx}
%\usepackage[svgnames]{xcolor}

%\linespread{1.25}

% \newtheorem{theorem}{Theorem}
% \newtheorem{corollary}{Corollary}
% \newtheorem{lemma}{Lemma}
% \newtheorem{invariant}{Invariant}
% \theoremstyle{definition}
% \newtheorem{definition}{Definition}

%\newcommand{\term}[1]{\textit{\textbf{#1}}}
% 
% \newcommand{\inthull}[1]{\textit{IH}(#1)}
% \newcommand{\angletuple}[1]{\langle #1 \rangle}
% \newcommand{\parentuple}[1]{(#1)}
% \newcommand{\floor}[1]{\left\lfloor #1 \right\rfloor}
% \newcommand{\ceiling}[1]{\left\lceil #1 \right\rceil}
% \newcommand{\ttfcn}[2]{\texttt{#1}\left(#2\right)}
% \newcommand{\fcnsign}[1]{\ttfcn{sign}{#1}}
% \newcommand{\fcnccw}[1]{\ttfcn{ccw}{#1}}
% \newcommand{\detrm}[1]{\begin{vmatrix}#1\end{vmatrix}}

% \newcommand{\tightoverset}[2]{
%     \mathop{#2}\limits^{\vbox to -.8ex{\kern-0.7ex\hbox{$#1$}\vss}}
% }
% 
% \newlength{\mOLineLength}
% \newcommand{\mOLine}[1]{
%     \tightoverset{
%     \setlength{\mOLineLength}{\widthof{#1}}
%     \begin{tikzpicture}
%         \draw [-to new, arrow head = 1.6pt, line width=0.5pt] (0, 0) --
%         +(0.8\mOLineLength, 0);
%     \end{tikzpicture}
%     }{\mathrm{#1}}
% }
% 
% \newcommand{\mRay}[1]{
%     \tightoverset{
%     \setlength{\mOLineLength}{\widthof{#1}}
%     \begin{tikzpicture}
%         \draw [* new-to new, arrow head = 1.6pt, line width=0.5pt]
%         (0, 0) -- +(0.8\mOLineLength, 0);
%         %\filldraw (0,0) circle (0.6pt);
%         %\draw [-angle 45 new, arrow head = 2.75pt, line width=0.4pt] (0, 0) --
%         %+(0.8\mOLineLength, 0);
%     \end{tikzpicture}
%     }{\mathrm{#1}}
% }
% 
% \newcommand{\mSeg}[1]{
%     \tightoverset{
%     \setlength{\mOLineLength}{\widthof{#1}}
%     \begin{tikzpicture}
%         \draw [* new-* new, arrow head = 1.6pt] (0, 0) -- +(0.8\mOLineLength,
%         0);
%     \end{tikzpicture}
%     }{\mathrm{#1}}
% }
% 
\newcommand{\mR}[1]{\mathbb{R}^{#1}}
\newcommand{\mZ}[1]{\mathbb{Z}^{#1}}
\newcommand{\mPoint}[1]{\mathrm{#1}}
\newcommand{\mPt}[1]{\mPoint{#1}}
\newcommand{\mVector}[1]{\mathbf{#1}}
\newcommand{\mVc}[1]{\mVector{#1}}
%\newcommand{\mSeg}[1]{\overline{\mathrm{#1}}}
%\newcommand{\mRay}[1]{\tightoverset{\rightharpoonup}{\mathrm{#1}}}
%\newcommand{\mRay}[1]{\tightoverset{\mapsto}{\mathrm{#1}}}
%\newcommand{\mOLine}[1]{\tightoverset{\testarrow}{\mathrm{#1}}}
%\newcommand{\mULine}[1]{\tightoverset{\leftrightarrow}{\mathrm{#1}}}
\newcommand{\mMatrix}[1]{\mathbf{#1}}
\newcommand{\mMt}[1]{\mMatrix{#1}}
\newcommand{\mPlane}[1]{\mathrm{#1}}
\newcommand{\mPolygon}[1]{\mathrm{#1}}
\newcommand{\mPolytope}[1]{\mathrm{#1}}
\newcommand{\mSet}[1]{\mathrm{#1}}
\newcommand{\mField}[1]{\mathrm{#1}}
\newcommand{\mCompound}[1]{\mathrm{#1}}

%==============================================================================

\begin{document}

%\title{Computing the integer hull of a simplicial cone in three dimensions}
\title{Computing a unimodular basis containing a given vector}
\author{Clinton Freeman}

\maketitle

\section{Introduction}

This document describes an algorithm that solves the following problem:
given a vector $\mVc{v} \in \mZ{3}$, compute an unimodular basis $\mMt{B}$ for
$\mZ{3}$ that has $\mVc{v}$ as one of the three basis vectors. In order to
precisely specify the input, output, and running time of the algorithm, I
provide a good deal of background material on linear algebra, models of
computation, and the computational complexity of basic arithmetic and algebraic
operations. 

\section{Background}

\subsection{Model of computation}

\subsection{Linear algebra}

\section{Algorithm description and analysis}

\begin{algorithm}
\caption{Constructing a unimodular basis that contains a specified vector}
\label{alg:UnimodularBasisWithVec}
\begin{algorithmic}[1]
    \vspace{0.75em}
    \Require{Integer vector $\mVector{v} \in \mR{3}$} 
    \Ensure{Unimodular basis $\mMatrix{B}$}
    \vspace{0.75em}
    \Procedure{UnimodularBasisWithVec3}{$\mVector{v}$}
    \State $\mMatrix{B} := \textsc{identity}$
    %\State $\mMatrix{B}_{i,i} := \mTernary{v_i < 0}{-1}{+1}$ 
    \While{$\mVector{v}$ is not a row of $\mMatrix{B}$}
        \State $\mVector{v}^\prime := \mVector{v} \cdot \mMatrix{B}^{-1}$
        \State $\mVector{v}^\prime_{\text{min}} :=
        \min(\mVector{v}^\prime)$, $\mVector{v}^\prime_{\text{mid}} :=
        \text{mid}(\mVector{v}^\prime)$, $\mVector{v}^\prime_{\text{max}} :=
        \max(\mVector{v}^\prime)$
        \State $\mMatrix{B}_{\text{min}, i} := \mMatrix{B}_{\text{min}, i} +
        \mMatrix{B}_{\text{mid}, i}$
        \State $\mMatrix{B}_{\text{min}, i} := \mMatrix{B}_{\text{min}, i} +
        \mMatrix{B}_{\text{max}, i}$
        \State $\mMatrix{B}_{\text{mid}, i} := \mMatrix{B}_{\text{mid}, i} +
        \mMatrix{B}_{\text{max}, i}$
    \EndWhile
    \State \Return $\mMatrix{B}$
    \EndProcedure 
\end{algorithmic}
\end{algorithm}

\end{document}
