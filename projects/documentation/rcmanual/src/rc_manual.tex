%==============================================================================
% @author Clinton Freeman
% @date 12/18/2013
%==============================================================================

\documentclass[oneside]{memoir}   

\usepackage{rc_style} 

\begin{document} 

%==============================================================================
% Cover page
%==============================================================================
   
\frontmatter

\pagenumbering{gobble}
\thispagestyle{empty}
   
\textbf{{\Huge RationalCAD}}

\clearpage

%==============================================================================
% Table of Contents
%==============================================================================

\pagenumbering{roman}

\tableofcontents* 
  
%==============================================================================
% Introduction
%==============================================================================

\mainmatter  
\pagenumbering{arabic}
     
\chapterstyle{freeman}
\chapter{Introduction}         

Hello and welcome to the RationalCAD manual! This document is intended to
catalogue the solutions to technical challenges faced by the software. It brings
together relevant background material, discusses both theoretical and practical 
issues, and provides explanations of concrete implementations in C++.   

\section{Audience}  
 
\section{General considerations} 
 
\subsection{Mathematical notation}
   
\begin{tabularx}{\textwidth}{llX}
\toprule 
Entity & Notation & Description \\  
\midrule
%Set & derp & derp \\
Set                         & $\{\square,\square,\square\}$                   & 
Curly braces surrounding a list denote a set of objects. \\

Tuple                       & $(\square,\square,\square)$                     & 
Parentheses surrounding a list denote a sequence of objects. \\

Elementary set              & $\mathbb{N, Z, Q, R}$                           & 
Naturals, integers, rationals, and reals. \\               

Elementary value            & $\alpha, \beta, a, b$                           &  
A value in some set. \\   
    
Point                       & $\mPt{p}, \mPt{q}, \mPt{r}, \mPt{s}$            & 
An element of some space such as $\mR{3}$. \\
    
Vector                      & $\mVc{t}, \mVc{u}, \mVc{v}, \mVc{w}$            &
A vector in some space such as $\mR{3}$. \\
 
Matrix                      & $\mMt{A}, \mMt{B}, \mMt{C}, \mMt{D}$            & 
description \\
 
Line (oriented)             & $\mOLine{ab}, \mOLine{cd}, \mOLine{A}, \mOLine{B}$        
& description \\  
%                
% %Line (unoriented)           & $\mULine{ab}, \mULine{pq}$                      &
% %description \\ 
 
Line segment                & $\mSeg{ab}, \mSeg{cd}, \mSeg{A}, \mSeg{B}$                         
& description \\  
   
Ray                         & $\mRay{ab}, \mRay{cd}, \mRay{A}, \mRay{B}$                               
& description \\           
    
Compound object             & $\mPlane{P}, \mPolygon{Q}, \mPolytope{R},
\mPolytope{S}$ & description \\                  

%Matrix transpose            & $a$ & description \\
%Matrix inverse              & $a$ & description \\
%Tuple                       & $a$ & description \\
%Determinant                 & $a$ & description \\
%Space (linear etc)          & $a$ & description \\ 
%Space (reals)               & $a$ & description \\      
%Dot (inner) product         & $a$ & description \\  
%Cross product               & $a$ & description \\ 
%Tensor (outer) product      & $a$ & description \\
 
%Vector length               & $a$ & description \\     
%Scalar                      & $a$ & description \\
%Angle                       & $a$ & description \\
%Unit vector                 & $a$ & description \\
%Perpendicular vector        & $a$ & description \\
%Parallel vector             & $a$ & description \\

\bottomrule          
\end{tabularx}   

\section{Branding and logo design}

\subsection{Font}

\subsection{Colors}

\subsection{Splash screen}

Splash screens are typically used by particularly large applications to notify
the user that the program is in the process of loading. They provide feedback
that a lengthy process is underway. Occasionally, a progress bar within the
splash screen indicates the loading progress. A splash screen disappears when
the application's main window appears. Splash screens typically serve to enhance
the look and feel of an application or web site, hence they are often visually
appealing. They may also have animations, graphics, and sound. (cite wiki)

\section{Coding style}

one true brace style, google style guide, etc

% \chapter{Basics of affine geometry}
% \section{Affine space}
% \section{Examples of affine spaces}
% \section{Chasles's identity}
% \section{Affine combinations, barycenters}
% \section{Affine subspaces}
% \section{Affine independence and affine frames}
% \section{Affine maps}
% \section{Affine groups}
% \section{Affine geometry: a glipse}
% \section{Affine hyperplanes}
% \section{Intersection of affine spaces}
% \section{Problems}

\chapter{Intersection computation}

\section{Sphere-sphere} 


Since the direction of each segment is allowed to vary, the start position and
length define the center and radius of respective spheres. The solution set may
be represented as the intersection of these spheres, which is one of the
following:
\begin{enumerate}
  \item The empty set $\emptyset$, which occurs in three cases. In the first
  case, the spheres are too far apart to touch, given by the conditions
  \begin{align} 
  \|\mPt{p_0}-\mPt{p_1}\| &> l_0,\text{ or} \\
  \|\mPt{p_0}-\mPt{p_1}\| &> l_1. 
  \end{align}
  In the second and third cases, one sphere is fully enclosed by the other,
  given by the conditions 
  \begin{align}
  \|\mPt{p_0}-\mPt{p_1}\| + l_0 &< l_1,\text{ or} \\
  \|\mPt{p_0}-\mPt{p_1}\| + l_1 &< l_0.
  \end{align}
  \item A single point $\mPt{q}$, which occurs when the distance between the two
  centers exactly equals the sum of the radii
  \begin{equation}
  \|\mPt{p_0}-\mPt{p_1}\|=l_0+l_1. 
  \end{equation}
  First observe that $\mPt{q}$ is located along the vector connecting the
  centers. We may use this fact to construct $\mPt{q}$ with a number of
  different algebraic expressions -- we should choose the expression of
  lowest degree in order to minimize its precision requirement. For example, a
  straightforward way is to form the vector $\mVc{v} = \mPt{p_0}-\mPt{p_1}$,
  normalize $\mVc{v}$, and construct $\mPt{q} = \mPt{p_1}+l_1\mVc{v}$. However,
  we can do better by simply computing $\mPt{q}$ as
  \begin{equation}
  \mPt{q} = \left(\frac{l_0+l_1}{l_1}\right)\mVc{v}.
  \end{equation}
  \item A circle $\mSet{C}$, which occurs when
  \begin{equation}
  \|\mPt{p_0}-\mPt{p_1}\|<l_0+l_1. 
  \end{equation}
  To begin, we may write $\mSet{C}$ precisely as the set of points 
  \begin{equation}\label{eq:circle_set}
  \mSet{C} = \left\{\mPt{q} \in \mR{3} : (\mPt{q} - \mPt{p_0})\cdot(\mPt{q} -
  \mPt{p_0})= l_0^2\text{ and } (\mPt{q} - \mPt{p_1})\cdot(\mPt{q} -
  \mPt{p_1})= l_1^2\right\}.
  \end{equation}
  For programming purposes, we would prefer a different representation. For
  simplicity, we can let $\mCompound{C}$ be the tuple $\mCompound{C} = (\mPt{c},
  r, \mVc{n})$, where $\mPt{c}$ is its center in $\mR{3}$, $\mVc{n}$ is normal
  to its supporting plane, and $r$ is its radius. We may compute this tuple from
  equation~\eqref{eq:circle_set} as follows. Intuitively, the normal $\mVc{n}$
  should be the normalized vector from one center to the other, $\mVc{n} =
  \mPt{p_0}-\mPt{p_1}$. 
  
  \item A sphere $\mCompound{S}$, when (L0 == L1) and (P0 == P1).
\end{enumerate}


\backmatter

\appendix
\chapter{Linear algebra}
 
\section{Vector spaces}

\begin{definition} 
A \term{vector space} over a field $\mField{F}$ is a set $\mSet{V}$, equipped
with two binary operations, that together satisfy eight axioms. Elements of 
$\mField{F}$ are called \term{scalars} and elements of $\mSet{V}$ are called
\term{vectors}. Let $s$, and $t$ be scalars and $\mVector{a}$, $\mVector{b}$,
and $\mVector{c}$ be vectors. The first binary operation is vector addition, $+ :
\mSet{V} \times \mSet{V} \to \mSet{V}$, written as $\mVector{a} + \mVector{b} =
\mVector{c}$.
The second binary operation is scalar multiplication, $\cdot : \mField{F} \times
\mSet{V} \to \mSet{V}$, written as $t \cdot \mVector{a} = \mVector{b}$ or
simply $t\mVector{a} = \mVector{b}$. A vector space satisfies the following
axioms.

\vspace{1em}

\noindent\begin{tabularx}{\textwidth}{lX}
\toprule 
Description & Axiom \\  
\midrule
Vector addition is associative. & $\mVector a + (\mVector b + \mVector c) =
(\mVector a + \mVector b) + \mVector c$ \\

Vector addition is commutative. & $\mVector a + \mVector b = \mVector b +
\mVector a$ \\

Additive identity for vector addition. & $\exists \, \mVector{0}
\in \mSet{V} :
\mVector{a}+\mVector{0}=\mVector{a}$\\

Additive inverse for all vectors. & $\exists \, \mVector{b} \in
\mSet{V} : \mVector{a}+\mVector{b} = \mVector{0}$ \\

Scalar multiplication distributes over vector addition. & $t(\mVector a +
\mVector b) = t\mVector a + t\mVector b$ 
\\

Scalar multiplication distributes over scalar addition. & $(s + t)\mVector a =
s\mVector a + t\mVector a$
\\

Scalar multiplication is associative. & $s(t\mVector a) = (st) \mVector a$ \\

Multiplicative identity for scalar multiplication & $1\mVector a = \mVector a$ \\
\bottomrule
\end{tabularx}  
\end{definition}


% \begin{center}
% \begin{tabular}{@{\makebox[3em][l]{(\rownumber)\space}} cl}
% $\mVector a + (\mVector b + \mVector c) = (\mVector a + \mVector b) + \mVector c$ & $\forall \, \mVector a, \mVector b, \mVector c \in V$ \\
% $\mVector a + \mVector b = \mVector b + \mVector a$ & $\forall \, \mVector a, \mVector b \in V$ \\
% $\exists \, \mVector 0 \in V \enspace \mid \enspace \mVector a + \mVector 0 = \mVector a$ & $\forall \, \mVector a \in V$\\
% $\exists \, \mVector b \in V \enspace \mid \enspace \mVector a + \mVector b = \mVector 0$ & $\forall \, \mVector a \in V$ \\
% $t(\mVector a + \mVector b) = t\mVector a + t\mVector b$ & $\forall \, t \in F, \, \mVector a, \mVector b \in V$ \\
% $(s + t)\mVector a = s\mVector a + t\mVector a$ & $\forall \, s, t \in F, \, \mVector a \in V$ \\
% $s(t\mVector a) = (st) \mVector a$ & $\forall \, s, t \in F, \, \mVector a \in V$ \\
% $1\mVector a = \mVector a$ & $\forall \, \mVector a \in V$
% \end{tabular}
% \end{center}


\end{document}
