%==============================================================================
% @author Clinton Freeman
% @date 12/18/2013
%==============================================================================

\documentclass[oneside]{memoir}   

\usepackage{rc_style} 

\begin{document} 

%==============================================================================
% Cover page
%==============================================================================
   
\frontmatter

%\pagecolor{answerColor}
%\color{white}

\pagenumbering{gobble}
\thispagestyle{empty}

%\begin{mdframed}[style=answer]  
\noindent\textbf{{\fontsize{20pt}{48pt}\selectfont ROBUST GEOMETRIC COMPUTING}}  
    
\vspace{1.5em}

\noindent\textit{\fontsize{16pt}{24pt}\selectfont A handy guide to writing
reliable geometric software}
%\end{mdframed}   
%\vspace{10em} 
\par\vspace*{\fill}

\noindent\hfill{\fontsize{16pt}{24pt}\selectfont by CLINTON FREEMAN}  

\clearpage

%\pagecolor{white}
%\color{black}

%\newpage\null\thispagestyle{empty}\newpage

%==============================================================================
% Table of Contents 
%==============================================================================

\pagenumbering{roman}

\tableofcontents*   

\clearpage

%==============================================================================
% Preface
%==============================================================================

\chapter{Preface}  

I am writing this book because I have yet to find a single source that clearly
outlines what is to be done about geometric nonrobustness from both a
theoretical and practical point of view.

\vspace{2em}
\hfill\textbf{Clinton Freeman}

\hfill\textit{Chapel Hill, North Carolina}

\hfill January 2014

%==============================================================================
% Introduction
%==============================================================================

\mainmatter  
\pagenumbering{arabic}
     
\chapterstyle{freeman}

\chapter{Uncategorized}         
 
% \chapterprecishere{``Begin at the beginning,� the King said gravely, ``and go on
% till you come to the end: then stop."\par\raggedleft--- \textup{Lewis Carroll}, Alice in Wonderland}

%\epigraph{``Begin at the beginning," the King said gravely, ``and go on till
% you come to the end: then stop."}{--- \textup{Lewis Carroll}, Alice in Wonderland}
 

%\section{Audience}   
 
\section{General considerations} 
 
\subsection{Mathematical notation}
   
\begin{tabularx}{\textwidth}{llX}
\toprule 
Entity & Notation & Description \\  
\midrule
%Set & derp & derp \\
Set                         & $\{\square,\square,\square\}$                   & 
Curly braces surrounding a list denote a set of objects. \\

Tuple                       & $(\square,\square,\square)$                     & 
Parentheses surrounding a list denote a sequence of objects. \\

Elementary set              & $\mathbb{N, Z, Q, R}$                           & 
Naturals, integers, rationals, and reals. \\               

Elementary value            & $\alpha, \beta, a, b$                           &  
A value in some set. \\   
    
Point                       & $\mPt{p}, \mPt{q}, \mPt{r}, \mPt{s}$            & 
An element of some space such as $\mR{3}$. \\
    
Vector                      & $\mVc{t}, \mVc{u}, \mVc{v}, \mVc{w}$            &
A vector in some space such as $\mR{3}$. \\
 
Matrix                      & $\mMt{A}, \mMt{B}, \mMt{C}, \mMt{D}$            & 
description \\
 
Line (oriented)             & $\mOLine{ab}, \mOLine{cd}, \mOLine{A}, \mOLine{B}$        
& description \\  
%                
% %Line (unoriented)           & $\mULine{ab}, \mULine{pq}$                      &
% %description \\ 
 
Line segment                & $\mSeg{ab}, \mSeg{cd}, \mSeg{A}, \mSeg{B}$                         
& description \\  
   
Ray                         & $\mRay{ab}, \mRay{cd}, \mRay{A}, \mRay{B}$                               
& description \\           
    
Compound object             & $\mPlane{P}, \mPolygon{Q}, \mPolytope{R},
\mPolytope{S}$ & description \\                  

%Matrix transpose            & $a$ & description \\
%Matrix inverse              & $a$ & description \\
%Tuple                       & $a$ & description \\
%Determinant                 & $a$ & description \\
%Space (linear etc)          & $a$ & description \\ 
%Space (reals)               & $a$ & description \\      
%Dot (inner) product         & $a$ & description \\  
%Cross product               & $a$ & description \\ 
%Tensor (outer) product      & $a$ & description \\
 
%Vector length               & $a$ & description \\     
%Scalar                      & $a$ & description \\
%Angle                       & $a$ & description \\
%Unit vector                 & $a$ & description \\
%Perpendicular vector        & $a$ & description \\
%Parallel vector             & $a$ & description \\

\bottomrule          
\end{tabularx}   

\subsection{Pseudocode}

\section{Geometric and solid modeling}

Geometric and solid modeling are processes by which a user may employ a computer
to design virtual geometric shapes. Another term for these processes is
computer-aided geometric design. The purpose may vary greatly, but we can
classify broadly as: creating a virtual object to live in a virtual world,
creating a virtual object to be manufactured to live in the real world, modeling
a real object, and \ldots The difference between geometric and solid modeling is
\ldots

%purpose of geometric modeling
%purpose of solid modeling
%difference between geometric and solid modeling
computer aided design

\section{Software reliability}

robust
reliable
software reliability
probability


\section{Polyhedra}

Our primary objects of study are convex polytopes and their corresponding
integer hulls. Below, we define both terms and bound the complexity of integer
hulls in 3-space by specializing a general bound for fixed dimensions.

In $\mR{d}$, let $S^d$ and $B^d$ denote the unit sphere and unit ball,
respectively, and for a set $X \subseteq \mR{d}$, let $\text{cl}X$ denote
the closure of $X$. A $k$\emph{-face} (or $k$\emph{-cell}) is subset of
$\mR{d}$ that is homeomorphic to $B^k$. A $d$-dimensional \emph{cell
complex} is a disjoint union of $k$-faces for $k=0,1,2,\ldots,d$ in which, for
any faces $a$ and $b$, the intersection of $\text{cl}a$ and $\text{cl}b$ is
either a union of faces or the empty set.

A \emph{convex polyhedron} $P$ is a nonempty intersection of a finite number of
closed half-spaces in $\mR{d}$. A \emph{convex polytope} is a convex
polyhedron whose volume is bounded. The reader should assume the words
polyhedron or polytope by themselves refer to a convex object unless otherwise
stated. The boundary of a polytope $P$, denoted $\partial P$, is a 2-dimensional
cell complex, and we call 0-faces \emph{vertices}, 1-faces \emph{edges}, and
2-faces \emph{facets}. We use the quad-edge data
structure~\cite{guibas1985primitives} to store and manipulate polytope
boundaries.

\subsection{Integer hulls and their complexity in 3-space}

As mentioned in the introduction, the \emph{integer hull} $\mIntHull{P}$ of a
polyhedron $P$ is the convex hull of all integer points contained by $P$. The
combinatorial complexity of $\mIntHull{P}$ is the sum of its vertices, edges,
and facets. In order to state an upper bound for the complexity of
$\mIntHull{P}$, we first define the \emph{size of an inequality}. For a rational
number $r = p/q$ with $p$ and $q$ relatively prime integers, let
$\text{size}(r) = 1+\ceiling{\log_2(|p|+1)}+\ceiling{\log_2(|q|+1)}$, and for a
rational vector $a$, $\text{size}(a) = n+\sum_{j=1}^n \text{size}(a_j)$. The
\emph{size of an inequality} $a^Tx \leq \beta$ is thus given by
$1+\text{size}(a)+\text{size}(\beta)$ and is an indication of the number of bits
necessary to encode it as a binary string.

\begin{theorem}
For a convex polyhedron $P \subset \mR{3}$ defined by $m$ inequalities
of size at most $\varphi$, the complexity of the integer hull $\mIntHull{P}$ is
$O(m^3\varphi^2)$.
\end{theorem}
\begin{proof}
Cook et.al. give an upper bound on the number of vertices of integer hulls in
$\mR{d}$~\cite{cook1992integer}. For fixed $d$, the bound is
$O(m^d\varphi^{d-1})$, so when $d=3$ we have $O(m^3\varphi^2)$. By Euler's
formula for planar graphs, we know that the complexity of a convex polyhedron in
three space is linear in the number of vertices~\cite{dutchBook}. Thus, we can
conclude that the overall complexity of $\mIntHull{P}$ shares the
$O(m^3\varphi^2)$ bound. 
\end{proof}

\subsection{Flattening planes}

In $\mR{3}$, we have the standard basis vectors $\{\mVector{i}, \mVector{j},
\mVector{k}\}$. These 3 standard vectors define what we may call the 3 standard
planes, $\{\hat{IJ}, \hat{IK}, \hat{JK}\}$. Intuitively, these planes -- taken
together -- divide 3-space into 8 octants. Given a plane $\Gamma \subset
\mR{3}$, how can we compute a transformation matrix that makes $\Gamma$
parallel to one of the three standard planes?

The basic idea is to find two linearly independent vectors that span $\Gamma$,
and make these two of our new basis vectors, combined with a third vector formed
from their cross product.

\subsection{Constructing unimodular bases}

\begin{algorithm}
\caption{Constructing a unimodular basis that contains a specified vector}
\label{alg:UnimodularBasisWithVec}
\begin{algorithmic}[1]
    \vspace{0.75em}
    \Require{Integer vector $\mVector{v} \in \mR{3}$} 
    \Ensure{Unimodular basis $\mMatrix{B}$}
    \vspace{0.75em}
    \Procedure{UnimodularBasisWithVec3}{$\mVector{v}$}
    \State $\mMatrix{B} := \textsc{identity}$
    \State $\mMatrix{B}_{i,i} := \mTernary{v_i < 0}{-1}{+1}$ 
    \While{$\mVector{v}$ is not a row of $\mMatrix{B}$}
        \State $\mVector{v}^\prime := \mVector{v} \cdot \mMatrix{B}^{-1}$
        \State $\mVector{v}^\prime_{\text{min}} :=
        \min(\mVector{v}^\prime)$, $\mVector{v}^\prime_{\text{mid}} :=
        \text{mid}(\mVector{v}^\prime)$, $\mVector{v}^\prime_{\text{max}} :=
        \max(\mVector{v}^\prime)$
        \State $\mMatrix{B}_{\text{min}, i} := \mMatrix{B}_{\text{min}, i} +
        \mMatrix{B}_{\text{mid}, i}$
        \State $\mMatrix{B}_{\text{min}, i} := \mMatrix{B}_{\text{min}, i} +
        \mMatrix{B}_{\text{max}, i}$
        \State $\mMatrix{B}_{\text{mid}, i} := \mMatrix{B}_{\text{mid}, i} +
        \mMatrix{B}_{\text{max}, i}$
    \EndWhile
    \State \Return $\mMatrix{B}$
    \EndProcedure 
\end{algorithmic} 
\end{algorithm}

\begin{lemma}
Given an integer vector $\mVector{v}$ with $\gcd(\mVector{v}) = 1$, Algorithm
\ref{alg:UnimodularBasisWithVec} computes a unimodular basis matrix with one row
equal to $\mVector{v}$ in $O(\log_\phi \varphi)$ time.
\end{lemma}
\begin{proof}
We have several proof obligations. First, we show that each iteration
computes a new basis that is closer to containing $\mVector{v}$. Next, we
show that output matrix $\mMatrix{B}$ is unimodular. Lines 2 and 3
establish the invariant that $|\mMatrix{B}| = \pm 1$. The loop beginning on line
4 maintains this invariant since $\mMatrix{B}$ is only modified with row
operations on lines 7-9, and row operations do not alter the determinant.
\end{proof}

polyhedron
polytope
polygon
polygonal chain
triangle
tetrahedron
simplex
cone
simplicial cone
mesh
simplicial mesh
subdivision
complex
simplicial complex
cross product
dot product
inner product
inner product space
projection
point
vertex
edge
face
line segment
line
ray
plane
hyperplane
open set
closed set
topology
homeomorphic
homology
cohomology
isomorphic
hausdorff distance
norm
euclidean distance
euclidean norm
manhattan distance
metric
taxicab metric
taxicab geometry
space
genus
Euler's formula
convex set
convex hull
integer hull
linear programming
simplex algorithm
integer programming
quadratic programming
linear inequality
halfspace
naturals
reals
integers
rationals
algebraic numbers
square root
kth root
floating point numbers
arithmetic operations
relative error
absolute error
tolerance
interval arithmetic
sin/cos/pi/e
transcendental numbers
numerical analysis
floor
ceiling
mod
logarithm
exponentiation
factorial
greatest common denominator
euclid's algorithm
extended euclid
diophantus
linear diophantine equations 
clipping a polytope with a plane
checking a polytope for convexity
boolean operations
rotations
rotating a collection of polytopes
CSG
Brep
high level geometric rounding
snap rounding
surface
boundary
manifold
winding order
predicate
construction
orientation predicate
matrix
matrix multiplication
matrix addition
matrix subtraction
matrix inverse
unimodular matrix
unimodular transformation
lattice
staircase


minkowski sum
hermann minkowski
convex decomposition
binary space partition
k-d tree
big-o notation
computational complexity
NP-hard
NP-complete
collision detection
axis
axis-aligned bounding box
oriented bounding box
separating axis theorem
circle
sphere
ball
quadric
cylinder
capsule


\section{Models of computation}

concrete machines
BSS
RAM
Real RAM
turing machine
straight-line program
decision-tree model
pseudocode


% \epigraph{``Begin at the beginning," the King said gravely, ``and go on till you 
% come to the end: then stop."}{--- \textup{Lewis Carroll}, Alice in Wonderland}

The following uses the \textbf{bitwise computation} model. Let $O_b$ denote
big-oh notation in this model.

\section{Complexity of arithmetic operations on integers}

\subsection{Addition}

Integer addition $+ : \mathbb{Z} \times \mathbb{Z} \mapsto \mathbb{Z}$ takes as
input two 


\begin{table}[hr]
\begin{tabularx}{\textwidth}{lX}
\toprule
Operation & Time complexity \\
\midrule
Addition & $O_b(\beta)$ \\
Subtraction & $O_b(\beta)$ \\
Multiplication & $O_b(\beta \log \beta \log \log \beta)$ \\
Division & $O_b(\beta \log \beta \log \log \beta)$ \\ 
\bottomrule
\end{tabularx}
\end{table}


rational multiplication
rational division
rational addition
rational subtraction

\section{Complexity of matrix operations}

\begin{table}[hr]
\caption{Time complexity of basic arithmetic operations on two $n \times n$
matrices.}
\begin{tabular}{ll}
\toprule 
Operation & Time complexity \\
\midrule
Multiplication & $O(n^{2.376})$ \\
Inverse & $O(n)$ \\
Determinant & $O_b(n \log n \log \log n)$ \\
\bottomrule
\end{tabular}
\end{table}

\chapter{Surface-surface intersection}

\section{Degree and genus of algebraic curves}

Katz and Sederberg presents formula for computing the degree and genus of the
intersection curve between two surfaces \cite{katz1988genus}. They show that for
two generic tensor product surface patches of parametric degree $m_1 \times m_2$
and $n_1 \times n_2$ respectively, the intersection curve is of degree
$4m_1m_2n_1n_2$ and generally of genus $8m_1m_2n_1n_2 -
2m_1m_2(n_1+n_2)-2n_1n_2(m_1+m_2)+1$. Bicubic patches (i.e. $m_1 = m_2 = 3$),
are often used in geometric modeling systems. Two such patches would intersect
in a curve of degree 324 and of genus 433. They explain that only curves whose
genus is zero can be expressed parametrically, curves whose genus is one or two
can be expressed using formula involving square roots, and curves whose genus is
greater than two possess no such parameterizations.

\section{B-splines}

B-splines are piecewise polynomial functions formed from linear combinations of
B-spline basis functions. These basis functions are in turn specified by a knot
vector, which is a monotonically increasing sequence of real numbers. The knot
vector defines the basis functions' support interval in parameter space. The
formula commonly used to compute the basis functions from the knot vector is
known as the Cox-de Boor recursion formula.

Although a more intuitive geometric way of deriving B-splines involves repeated
convex combinations of control points, the vector space point of view is more
important for considering tensor product surfaces and T-splines. For T-splines,
being able to derive the basis functions independently of the control points is
necessary since they are constructed with unique knot vectors for each control
point. Since there is only one control point per set of basis functions, we are
unable to use the repeated convex combination construction, and must rely on the
Cox-de Boor recurrence.

\section{Tensors, the tensor product, and tensor product surfaces}

Tensors (from the Latin word \emph{tendere}, ``to stretch'') came into use
around the turn of the 19\textsuperscript{th} century by mathematicians working
in differential geometry, perhaps most notably by Ricci and Levi-Civita. One
most often finds tensors in physics literature, where they can be used to
succinctly describe physical concepts such as stress and strain on geometric
objects. Einstein used tensors extensively in his theory of general relativity,
which helped give tensors more widespread acceptance. He also introduced an
eponymous index notation to simplify the sometimes unwieldy expressions. This
notation is now standard practice when working with tensors.

A natural way for computer scientists to think about tensors is to view them as
multidimensional arrays paired with a transformation law which describes how the
entries respond to a change of basis. Each element in the array is called a
component, and represents a value of the tensor with respect to a particular
basis. If we perform a change of basis with a transformation matrix $R$, the
transformation law just describes how each component is mapped to the new basis.
That is to say, whether the component will be multiplied by $R$ or by $R^{-1}$.
Components which are multiplied by $R$ are said to be covariant and their index
appears as a subscript, and components which are multiplied by $R^{-1}$ are said
to be contravariant and their index appears as a superscript. We can also talk
about the order of a tensor, which is the number of indices you need to specify
a particular element in the array. The familiar concepts of scalars, vectors,
and matrices are subsumed under the definition of a tensor; they are order 0, 1,
and 2 tensors, respectively.

There are a number of ways to view tensor products. When talking about ``tensor
product surfaces,'' we are referring to taking a product of vector spaces. The
first vector space is specified by the $U$ knot vector, which generates a
corresponding set of basis functions. The set of all linear combinations of
these basis functions is the vector space. Likewise, another vector space is
generated from the second knot vector $V$. The tensor product is formed from the
Cartesian product that is modified with equivalence relations.

\section{Sphere-sphere} 


Since the direction of each segment is allowed to vary, the start position and
length define the center and radius of respective spheres. The solution set may
be represented as the intersection of these spheres, which is one of the
following:
\begin{enumerate}
  \item The empty set $\emptyset$, which occurs in three cases. In the first
  case, the spheres are too far apart to touch, given by the conditions
  \begin{align} 
  \|\mPt{p_0}-\mPt{p_1}\| &> l_0,\text{ or} \\
  \|\mPt{p_0}-\mPt{p_1}\| &> l_1. 
  \end{align}
  In the second and third cases, one sphere is fully enclosed by the other,
  given by the conditions 
  \begin{align}
  \|\mPt{p_0}-\mPt{p_1}\| + l_0 &< l_1,\text{ or} \\
  \|\mPt{p_0}-\mPt{p_1}\| + l_1 &< l_0.
  \end{align}
  \item A single point $\mPt{q}$, which occurs when the distance between the two
  centers exactly equals the sum of the radii
  \begin{equation}
  \|\mPt{p_0}-\mPt{p_1}\|=l_0+l_1. 
  \end{equation}
  First observe that $\mPt{q}$ is located along the vector connecting the
  centers. We may use this fact to construct $\mPt{q}$ with a number of
  different algebraic expressions -- we should choose the expression of
  lowest degree in order to minimize its precision requirement. For example, a
  straightforward way is to form the vector $\mVc{v} = \mPt{p_0}-\mPt{p_1}$,
  normalize $\mVc{v}$, and construct $\mPt{q} = \mPt{p_1}+l_1\mVc{v}$. However,
  we can do better by simply computing $\mPt{q}$ as
  \begin{equation}
  \mPt{q} = \left(\frac{l_0+l_1}{l_1}\right)\mVc{v}.
  \end{equation}
  \item A circle $\mSet{C}$, which occurs when
  \begin{equation}
  \|\mPt{p_0}-\mPt{p_1}\|<l_0+l_1. 
  \end{equation}
  To begin, we may write $\mSet{C}$ precisely as the set of points 
  \begin{equation}\label{eq:circle_set}
  \mSet{C} = \left\{\mPt{q} \in \mR{3} : (\mPt{q} - \mPt{p_0})\cdot(\mPt{q} -
  \mPt{p_0})= l_0^2\text{ and } (\mPt{q} - \mPt{p_1})\cdot(\mPt{q} -
  \mPt{p_1})= l_1^2\right\}.
  \end{equation}
  For programming purposes, we would prefer a different representation. For
  simplicity, we can let $\mCompound{C}$ be the tuple $\mCompound{C} = (\mPt{c},
  r, \mVc{n})$, where $\mPt{c}$ is its center in $\mR{3}$, $\mVc{n}$ is normal
  to its supporting plane, and $r$ is its radius. We may compute this tuple from
  equation~\eqref{eq:circle_set} as follows. Intuitively, the normal $\mVc{n}$
  should be the normalized vector from one center to the other, $\mVc{n} =
  \mPt{p_0}-\mPt{p_1}$. 
  
  \item A sphere $\mCompound{S}$, when (L0 == L1) and (P0 == P1).
\end{enumerate}


\backmatter

\appendix
\chapter{Linear algebra}

vector
vector space
change of basis
basis
frame
determinant
derterminant computation, 2x2 3x3 4x4, nxn
transpose

\section{Vector spaces}

\begin{definition} 
A \term{vector space} over a field $\mField{F}$ is a set $\mSet{V}$, equipped
with two binary operations, that together satisfy eight axioms. Elements of 
$\mField{F}$ are called \term{scalars} and elements of $\mSet{V}$ are called
\term{vectors}. Let $s$, and $t$ be scalars and $\mVector{a}$, $\mVector{b}$,
and $\mVector{c}$ be vectors. The first binary operation is vector addition, $+ :
\mSet{V} \times \mSet{V} \to \mSet{V}$, written as $\mVector{a} + \mVector{b} =
\mVector{c}$.
The second binary operation is scalar multiplication, $\cdot : \mField{F} \times
\mSet{V} \to \mSet{V}$, written as $t \cdot \mVector{a} = \mVector{b}$ or
simply $t\mVector{a} = \mVector{b}$. A vector space satisfies the following
axioms.

\vspace{1em}

\noindent\begin{tabularx}{\textwidth}{lX}
\toprule 
Description & Axiom \\  
\midrule
Vector addition is associative. & $\mVector a + (\mVector b + \mVector c) =
(\mVector a + \mVector b) + \mVector c$ \\

Vector addition is commutative. & $\mVector a + \mVector b = \mVector b +
\mVector a$ \\

Additive identity for vector addition. & $\exists \, \mVector{0}
\in \mSet{V} :
\mVector{a}+\mVector{0}=\mVector{a}$\\

Additive inverse for all vectors. & $\exists \, \mVector{b} \in
\mSet{V} : \mVector{a}+\mVector{b} = \mVector{0}$ \\

Scalar multiplication distributes over vector addition. & $t(\mVector a +
\mVector b) = t\mVector a + t\mVector b$ 
\\

Scalar multiplication distributes over scalar addition. & $(s + t)\mVector a =
s\mVector a + t\mVector a$
\\

Scalar multiplication is associative. & $s(t\mVector a) = (st) \mVector a$ \\

Multiplicative identity for scalar multiplication & $1\mVector a = \mVector a$ \\
\bottomrule
\end{tabularx}  
\end{definition}


% \begin{center}
% \begin{tabular}{@{\makebox[3em][l]{(\rownumber)\space}} cl}
% $\mVector a + (\mVector b + \mVector c) = (\mVector a + \mVector b) + \mVector c$ & $\forall \, \mVector a, \mVector b, \mVector c \in V$ \\
% $\mVector a + \mVector b = \mVector b + \mVector a$ & $\forall \, \mVector a, \mVector b \in V$ \\
% $\exists \, \mVector 0 \in V \enspace \mid \enspace \mVector a + \mVector 0 = \mVector a$ & $\forall \, \mVector a \in V$\\
% $\exists \, \mVector b \in V \enspace \mid \enspace \mVector a + \mVector b = \mVector 0$ & $\forall \, \mVector a \in V$ \\
% $t(\mVector a + \mVector b) = t\mVector a + t\mVector b$ & $\forall \, t \in F, \, \mVector a, \mVector b \in V$ \\
% $(s + t)\mVector a = s\mVector a + t\mVector a$ & $\forall \, s, t \in F, \, \mVector a \in V$ \\
% $s(t\mVector a) = (st) \mVector a$ & $\forall \, s, t \in F, \, \mVector a \in V$ \\
% $1\mVector a = \mVector a$ & $\forall \, \mVector a \in V$
% \end{tabular}
% \end{center}


\bibliographystyle{plain}  
\bibliography{rc_references} 

\end{document}

\endinput



\lettrine[lines=2, findent=0.25em, nindent=0em]{G}{eometric} objects on a
digital computer are composed of two types of data: numerical and combinatorial.
Examples of numerical data may include the Cartesian coordinates of a point in
3-space, the length of a line segment connecting two such points, or the angle
between two such line segments. Examples of combinatorial information may
include grouping two points as an edge, grouping a collection of edges as a
face, or grouping a collection of faces as a surface.

Geometric algorithms that operate on geometric objects are best thought of as
two types of operations: predicates and constructions. Predicates determine
relationships between objects. A predicate might determine if a point is to the
left, right, or is collinear with a line segment, determine if a point is
inside, outside, or on a circle, or determine if a line intersects a plane in
one, none, or infinitely many points. Constructions produce new geometric
objects from existing geometric objects. A construction might produce the
rotation of a point around an origin, produce the point of intersection between
two line segments, or produce an offset of an algebraic curve.

Geometric algorithms are typically designed and analyzed using the Real-RAM
model of computation~\cite{preparata1977convex}. In other words, these
algorithms assume that the numerical data in geometric objects are exact values
in $\mathbb{R}$ that can be stored and retrieved in constant time, and that
arithmetic involving these values is performed in constant time. From a
practical point of view, it may seem like an odd or frustrating decision to
assume access to infinite precision real arithmetic, given that digital
computers are finite objects. From a theoretical point of view, this is a
sensible choice given that for subsets of $\mathbb{R}$, such as the rational
numbers $\mathbb{Q}$, many fundamental geometric axioms no longer hold.

Geometric nonrobustness results from this unfortunate disconnect between
continuous theoretical formulations and the reality of discrete machine
implementation. In most instances, the numerical data composing geometric
objects is an approximation to a real value. Predicates that assume exact values
but are fed approximate values are liable to make incorrect determinations.
Constructions compound the situation by taking exact values and producing
approximate ones, or by taking approximate values and producing even coarser
approximations. In short, geometric nonrobustness is a problem wherein branching
decisions in geometric algorithms are predicated on approximate numerical
computations, leading to various forms of unreliability including ill-formed
output and software failure through crashing or infinite loops.

\section{Consequences of geometric nonrobustness}

Broadly speaking, the consequence of geometric nonrobustness is software
unreliability. That is, nonrobust geometric algorithms do not perform their
stated function for all inputs; they can enter into undefined states and produce
incorrect output. Algorithms that are theoretically deterministic are
practically probabilistic. Fortunately, software unreliability is a
well-established concept. By first equating geometric nonrobustness with
software unreliability, we can understand the former in terms of the latter.

The broad consequence of software unreliability can only be refined in reference
to a specific application. As we refine from broad to specific, we eventually
arrive at the real-world impact of the unreliable software. This abstract
process can not actually be carried out in full. Instead, we can sketch a
picture of how geometric nonrobustness manifests in real-world situations with
various types of data, including anecdotal documents, economic impact
assessments, and publically available industrial cost estimates. Below, I
provide examples of such data from two applications which make extensive use of
geometric computation: real-time video games and computer-aided design software.

The first situation involves the tradeoffs between network efficiency, collision
detection accuracy, and visual fidelity involved in choosing a numerical
representation for three dimensional position coordinates. In a technical log,
the lead engineer of video game developer id Software discusses how he chose to
balance each of these competing ends. The coordinates needed to be transmitted
over a network, so quantizing the floating point components was attractive from
an efficiency point of view. However, this resulted in erroneous collisions and
discernably discrete movement of game characters. As Carmack writes
\cite{carmack1999plan}, {\itshape ``Another aspect of the problem that wasn't
visible to the public was that the the fractional quantization of position could
cause the position to actually be inside a nearby solid when used for client
side prediction. The code had to check for this and try to correct the situation
by jittering the position in each of the possible directions it might have been
truncated from. This is a potential issue whenever there is any loss of
precision whatsoever in the server to client communication.'' }

The second situation involves the transfer of CAD geometry to analysis
algorithms. At a 1999 workshop on the mathematical foundations of CAD software,
members of academia and industry sought to address the fact that CAD systems
were still lacking in efficiency, reliability, and compatibility with subsequent
analysis tools such as computational fluid dynamics (CFD). Blame for the
incompatibility between CAD and analysis was laid directly at the feet of
geometric nonrobustness. As Farouki notes \cite{farouki1999closing}, {\itshape
``the overriding theme of the workshop, echoed in virtually all the position
papers, was the cumbersome and error-prone process of deriving satisfactory CFD
surface and volume meshes from CAD models. This problem is not the fault of the
meshing algorithms, but rather of the geometrical or topological errors and
inconsistencies that plague CAD models.'' }

Reliability is subject to cost-benefit analysis alongside other measures of
software quality: developers allocate their time and effort among many
alternatives in order to maximize utility for the end user. In real-time
graphics applications such as video games, where practitioners are known for
ruthlessly approximating to shave off clock cycles, software developers
routinely encounter the effects of nonrobustness. In the transfer of geometry
from CAD to analysis tools such as CFD, nonrobustness plays a central role as a
frustrating and costly problem.

\section{The state of robust geometric computing}

Today, practitioners faced with writing geometric code typically follow two
basic strategies. The first is to use floating-point representations, empirical
epsilon tolerancing, and various types of application-specific special case
handling. The strategy is to minimize the probability of failure, and when
failures occur, devise ways to ensure the failure is in some way graceful. The
second is to follow the Exact Geometric Computation (EGC) paradigm
\cite{yap1995exact}, in which algorithms use arbitrary precision arithmetic in
order to ensure predicate evaluations are correct. The EGC paradigm is embodied
in software libraries such as LEDA, CORE, and CGAL. CGAL is notable since users
can choose to parameterize algorithms with exact number types for completely
robust results, or to rely on floating-point for faster, but potentially
incorrect, results.

EGC is a very powerful framework when computing purely combinatorial
information. For example, convex hull algorithms are given a set of points, and
return another set of points chosen from the input set. EGC works well because
predicates only have to temporarily compute higher precision values, and the
numerical precision in the output is the same as in the input. EGC falls short
when computing repeated, or cascaded, constructions. For example, rigid body
rotations of objects increase the amount of precision required for exact
representation of rotated object. Users of CAD software apply these types of
operations over and over again when modeling geometry. This growth of precision
will quickly become prohibitive, and naive rounding can introduce
inconsistencies between numerical and combinatorial data.


The following is taken from Hopcroft and Ullman.

Given a problem, how do we find an efficient algorithm for its solution? Once we
have found an algorithm, how can we compare this algorithm with other algorithms
that solve the same problem? How should we judge the goodness of an algorithm?
Questions of this nature are of interest both to programmers and to
theoretically oriented computer scientists. In this book we shall examine
various lines of research that attempt to answer questions such as these.

In this chapter, we consider several models of a computer -- the random access
machine, the random access stored program machine, and the Turing machine. We
compare these models on the basis of their ability to reflect the complexity of
an algorithm, and derive from them several more specialized models of
computation, namely, straight-line arithmetic sequences, bitwise computations,
bit vector computations, and decision trees. Finally, in the last section of
this chapter we introduce a language called Pidgin Algol for describing
algorithms.



\lettrine[lines=2, findent=0.25em, nindent=0em]{H}{ello} and welcome to the
RationalCAD manual! This document is intended to catalogue the solutions to technical challenges faced by the software. It brings
together relevant background material, discusses both theoretical and practical 
issues, and provides explanations of concrete implementations in C++.   


\section{Branding and logo design}

\subsection{Font}

\subsection{Colors}

\subsection{Splash screen}
 
Splash screens are typically used by particularly large applications to notify
the user that the program is in the process of loading. They provide feedback
that a lengthy process is underway. Occasionally, a progress bar within the
splash screen indicates the loading progress. A splash screen disappears when
the application's main window appears. Splash screens typically serve to enhance
the look and feel of an application or web site, hence they are often visually
appealing. They may also have animations, graphics, and sound. (cite wiki)

\section{Coding style}

one true brace style, google style guide, etc

% \chapter{Basics of affine geometry}
% \section{Affine space}
% \section{Examples of affine spaces}
% \section{Chasles's identity}
% \section{Affine combinations, barycenters}
% \section{Affine subspaces}
% \section{Affine independence and affine frames}
% \section{Affine maps}
% \section{Affine groups}
% \section{Affine geometry: a glipse}
% \section{Affine hyperplanes}
% \section{Intersection of affine spaces}
% \section{Problems}


\chapter{Grid system}

\epigraph{``Begin at the beginning," the King said gravely, ``and go on till you 
come to the end: then stop."}{--- \textup{Lewis Carroll}, Alice in Wonderland}

Several objects and settings are relevant to specifying how the 2D grid is drawn:

\begin{itemize}
  \item QGLWidget: acts as a container and provides a GL drawing surface (API).
  \item glViewport: sets up the correspondence between the dimensions of the
  penultimate GL image and the GL drawing surface (API).
  \item Projection matrix: describes the region of space which will be included
  in the final image.
  \item Modelview matrix: induces various transformations on the grid vertices;
  we will need scaling and translation, but not rotation.
  \item Vertex coordinates: describes the untransformed position of gridlines in
  world space.
\end{itemize}

